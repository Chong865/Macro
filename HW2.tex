\documentclass{article}
\usepackage[utf8]{inputenc}
\usepackage{amsmath}
\usepackage{graphicx}
\title{Adv. Macroeconomics II Problem Set 2 Solution}
\author{Chong Wang}

\begin{document}
\maketitle
\section{Define a competitive equilibrium for this economy}
A Competitive Equilibrium consists of prices $\{p_{t}, w_{t}, r_{t}\}^{\infty}_{t=0}$ and allocations for the firm $\{k^{d}_{t}, l^{d}_{t}, y_{t}\}^{\infty}_{t=0}$ and the household $\{c_{t}, i_{t}, x_{t+1}, k^{s}_{t}, l^{s}_{t}\}^{\infty}_{t=0}$ such that
\begin{enumerate}
  \item Given prices $\{p_{t}, w_{t}, r_{t}\}^{\infty}_{t=0}$, the allocation of the representative firm $\{k^{d}_{t}, l^{d}_{t}, y_{t}\}^{\infty}_{t=0}$ solves:\\
  
  \[ \pi = \max\limits_{\{k^{d}_{t}, l^{d}_{t}, y_{t}\}^{\infty}_{t=0}} \sum_{t=0}^{\infty} p_{t}(y_{t}-r_{t}k^{d}_{t}-w_{t}l^{d}_{t}) \]
  \[ s.t.\quad y_{t} = z(k^{d}_{t})^\alpha(l^{d}_{t})^{1-\alpha}\quad \forall t\geq0\]
  \[ y_{t}, k^{d}_{t}, l^{d}_{t}\geq0\quad \forall t\geq0\]
  
  \item Given prices $\{p_{t}, w_{t}, r_{t}\}^{\infty}_{t=0}$, the allocation of the representative household $\{c_{t}, i_{t}, x_{t+1}, k^{s}_{t}, l^{s}_{t}\}^{\infty}_{t=0}$ solves:
  
   \[ \max\limits_{\{c_{t}, i_{t}, x_{t+1}, k^{s}_{t}, l^{s}_{t}\}^{\infty}_{t=0}} \sum_{t=0}^{\infty} \beta_{t}(\frac{(c_{t})^{1-\sigma}}{1-\sigma}-\chi \frac{(l^s_{t})^{1+\eta}}{1+\eta})\]
  \[ s.t.\quad \sum_{t=0}^{\infty}p_{t}(c_{t}+i_{t}) \leq \sum_{t=0}^{\infty}p_{t}(r_{t}k^{s}_{t} + w_{t}l^{s}_{t}) + \pi\]
  \[ x_{t+1} = (1-\delta)x_{t}+i_{t} \quad \forall t\geq0\]
  \[  0\leq l^{s}_{t}\leq1,\quad0\leq k^{s}_{t}\leq x_{t}\quad \forall t\geq0\]
  \[ c_{t}, x_{t+1}\geq0\quad \forall t\geq0\]
  \[ k_{0}\quad given\]
  
  \item Markets Clear
  \begin{equation}
    y_{t} = c_{t} + i_{t} \quad(Goods \:Market\:)
  \end{equation}
  \begin{equation}
  l^{d}_{t} = l^{s}_{t} = l_{t} \quad(Labor \:Market\:)
  \end{equation}
  \begin{equation}
  k^{d}_{t} = k^{s}_{t} =k_{t}\quad(Capital \:Market\:)
  \end{equation}
\end{enumerate}

\section{Find the steady state value for $\{c,l,k,r,w,y\}$}
Firm's problem: take FOC w.r.t $k_{t}$ and $l_{t}$
    \begin{equation}
    \begin{split}
    r_{t} &= z\alpha (k_{t})^{\alpha-1}(l_{t})^{1-\alpha}\\
    w_{t} &= z(1-\alpha)(k_t)^{\alpha} (l_t)^{-\alpha}\\
    \pi &= \sum_{t=0}^{\infty} p_{t}(z(k_{t})^\alpha(l_{t})^{1-\alpha}-r_{t}k_{t}-w_{t}l_{t})=0
    \end{split}
    \end{equation}
Household's problem: there is no gains from not fully using capital on $k_t$, so $k_t = x_t \quad \forall t\geq0 $. Then, solve the Lagrange function.
    \begin{equation}
    \begin{split}
    \mathcal{L}(\{c_t,k_t,l_t\}_{t=0}^{\infty},\lambda_t)=&\sum_{t=0}^{\infty} \beta ^t(\frac{c_t^{1-\sigma}}{1-\sigma}-\chi \frac{l_t^{1+\eta}}{1+\eta})\\
    &+\lambda_t(\sum_{t=0}^{\infty}p_t(zk_t^{\alpha}l_t^{1-\alpha}-c_t-k_{t+1}+(1-\delta)k_t)\\
    \frac{\partial \mathcal{L}}{\partial c_t} &=\beta^tc_t^{-\sigma}-\lambda_t p_t=0\\
    \frac{\partial \mathcal{L}}{\partial l_t} &=-\beta^t\chi l_t^{\eta}+\lambda_t p_tz(1-\alpha) k_t^{\alpha}l_t^{-\alpha}=0\\
    \frac{\partial \mathcal{L}}{\partial k_t}&=\lambda_t p_t(z\alpha k_t^{\alpha -1}l_t^{1-\alpha}+1-\delta)-\lambda_{t-1}p_{t-1}=0\\
    \frac{\partial \mathcal{L}}{\partial \lambda_t} &= zk_t^{\alpha}l_t^{1-\alpha}-k_{t+1}+(1-\delta)k_t - c_t = 0
    \end{split}
    \end{equation}
 In steady state, $c_t =c, k_t =k, l_t =l$. When we normalize $p_0 =1$, we get:
 \begin{equation}
    \begin{split}
    \lambda_t &= c\\
    p_t &= \beta^t\\
    c^{\sigma} l^{\eta}&= z(1-\alpha)k^{\alpha}l^{-\alpha}/\chi\\
    z\alpha k^{\alpha -1}l^{1-\alpha}&= 1/\beta -1 +\delta\\
    c&=zk^{\alpha}l^{1-\alpha}-\delta k
    \end{split}
    \end{equation}
 Now, solve for $\frac{k}{l}$ and $\frac{c}{l}$, then use them to solve the steady state variables:
  \begin{equation}
    \begin{split}
    Let\: M &= \frac{k}{l}=\left(\frac{z\alpha \beta}{1-\beta +\beta \delta}\right)^{\frac{1}{1-\alpha}} \\
    Let\: N &=\frac{c}{l}=z(\frac{k}{l})^{\alpha}-\delta \frac{k}{l}=zM^{\alpha}-\delta M \\
    c^{\sigma} l^{\eta} &= (lN)^{\sigma}l^{\eta}\\
    &= z(1-\alpha)(\frac{k}{l})^{\alpha}/\chi\\
    &\Rightarrow l^{\sigma+\eta} =\frac{z(1-\alpha)M^{\alpha}}{\chi N^{\sigma}}\\ &\Rightarrow l=\left(\frac{z(1-\alpha)M^{\alpha}}{\chi N^{\sigma}}\right)^{\frac{1}{\sigma+\eta}} \quad (*)
    \end{split}
    \end{equation}
With $l$ in steady state, we can solve for the other steady state variables:
  \begin{equation}
    \begin{split}
    k&=Ml\quad (*)\\
    c&=Nl\quad (*)\\
    y&=zk^{\alpha}l^{1-\alpha}=zM^{\alpha}l\quad (*)\\
    r&=\alpha zk^{\alpha-1}l^{1-\alpha}=\alpha zM^{\alpha -1}\quad (*)\\
    w&=(1-\alpha)zk^{\alpha}l^{-\alpha}=(1-\alpha)zM^{\alpha} \quad (*)
    \end{split}
    \end{equation}
 \section{Pose the planner’s dynamic programming problem. Write down the appropriate
Bellman equation.}
The social planner's problem is:
\begin{equation}
\begin{split}
w(k_{0}) &= \max\limits_{\{k_{t}, c_{t}, l_{t}\}^{\infty}_{t=0}} \sum_{t=0}^{\infty} \beta^t(\frac{(c_{t})^{1-\sigma}}{1-\sigma}-\chi \frac{(l^s_{t})^{1+\eta}}{1+\eta}) \\
    s.t. \quad z(k_t)^{\alpha}(l_t)^{1-\alpha} &= c_t+k_{t+1}-(1-\delta)k_t \quad \forall t\geq0\\
    c_t &\geq 0, k_t \geq 0, 0 \leq l_t \leq 1 \quad \forall t\geq0\\
     \quad {k}_{0}\:given
\end{split}
\end{equation}
Then, substitute $c_t$ using $k_t$ and turn it into Bellman equation:
\begin{equation}
\begin{split}
v(k_0) = \max\limits_{\begin{smallmatrix}0\leq k' \leq z(k)^{\alpha}(l)^{1-\alpha}+ (1-\delta)k\\ 0\leq l \leq 1\end{smallmatrix}} \{\frac{(z(k)^{\alpha}(l)^{1-\alpha}+ (1-\delta)k-k')^{1-\sigma}}{1-\sigma}-\chi \frac{(l)^{1+\eta}}{1+\eta}+\beta v(k')\}
\end{split}
\end{equation}
\section{Find $\chi$ such that $l_{ss}$ = 0.4}
Rewrite the equation $l_{ss}$ for $\chi$
\begin{equation}
\begin{split}
l_{ss}^{\sigma+\eta} =\frac{z(1-\alpha)M^{\alpha}}{\chi N^{\sigma}}\\
\chi = \frac{z(1-\alpha)M^{\alpha}}{l_{ss}^{\sigma+\eta} N^{\sigma}}
\end{split}
\end{equation}
Plug in $\alpha=1/3,z=1,\delta=2,\eta=1,l_{ss}=0.4$
\begin{equation}
\chi = 57.12
\end{equation}
\section{Solve the planner’ problem numerically using value function iteration. You must do it using:}
\subsection{Plain VFI}
n\_k = 100, Converging Iteration: 266, Time: 176s
\begin{figure}[!htbp]
\centering
  \includegraphics[scale=0.45]{HW2_5a_Value.png}
  \caption{Approximated Value Function}
  \label{fig:boat1}
\end{figure}
\pagebreak
\begin{figure}[!htbp]
\centering
  \includegraphics[scale=0.45]{HW2_5a_policy.png}
  \caption{Approximated Policy Function}
  \label{fig:boat1}
\end{figure}
\begin{figure}[!htbp]
\centering
  \includegraphics[scale=0.45]{HW2_5a_Euler.png}
  \caption{Approximated Error($\%$)}
  \label{fig:boat1}
\end{figure}
\subsection{Modified Howard’s Policy Iteration (you must choose the number of policy iterations)}
n\_k = 500, Converging Iteration: 19, Time: 33s
\pagebreak
\begin{figure}[!htbp]
\centering
  \includegraphics[scale=0.45]{HW2_5b_value.png}
  \caption{Approximated Value Function}
  \label{fig:boat1}
\end{figure}
\begin{figure}[!htbp]
\centering
  \includegraphics[scale=0.45]{HW2_5b_policy.png}
  \caption{Approximated Policy Function}
  \label{fig:boat1}
\end{figure}
\pagebreak
\begin{figure}[!htbp]
\centering
  \includegraphics[scale=0.45]{HW2_5b_Euler.png}
  \caption{Approximated Error($\%$)}
  \label{fig:boat1}
\end{figure}

\subsection{MacQueen-Porteus Bounds}
n\_k = 600, Converging Iteration: 17, Time: 20s
\begin{figure}[!htbp]
\centering
  \includegraphics[scale=0.45]{HW2_5c_value.png}
  \caption{Approximated Value Function}
  \label{fig:boat1}
\end{figure}
\begin{figure}[!htbp]
\centering
  \includegraphics[scale=0.45]{HW2_5c_policy.png}
  \caption{Approximated Policy Function}
  \label{fig:boat1}
\end{figure}
\begin{figure}[!htbp]
\centering
  \includegraphics[scale=0.45]{HW2_5c_Euler.png}
  \caption{Approximated Error($\%$)}
  \label{fig:boat1}
\end{figure}
\section{Use the solution to the planner’s problem to obtain the steady state value of $\{c, k, r,l,w, y\}$. (\underline{Compare estimation to solved steady state})}
\subsection{Capital decreases to 80$\%$ of its steady state value}
\begin{figure}[!htbp]
\centering
  \includegraphics[scale=0.45]{HW2_6a_k.png}
  \caption{SS capital decreases to 80$\%$ }
  \label{fig:boat1}
\end{figure}
\subsection{Productivity increases permanently by 5$\%$}
\pagebreak
\begin{figure}[!htbp]
\centering
  \includegraphics[scale=0.45]{HW2_6b_z.png}
  \caption{Productivity increases permanently by 5$\%$ }
  \label{fig:boat1}
\end{figure}
\section{Prove that the mapping used in Howard’s policy iteration algorithm is a contraction.}
I will prove the contraction by Blackwell's theorem.\\
\begin{equation}
\begin{split}
Tv^{n}(k) = \max\limits_{\begin{smallmatrix}0\leq k' \leq z(k)^{\alpha}(l)^{1-\alpha}+ (1-\delta)k\\ 0\leq l \leq 1\end{smallmatrix}} \{\frac{(z(k)^{\alpha}(l)^{1-\alpha}+ (1-\delta)k-k')^{1-\sigma}}{1-\sigma}-\chi \frac{(l)^{1+\eta}}{1+\eta}+\beta v^n(k')\}
\end{split}
\end{equation}
- First, we assume $\frac{(z(k)^{\alpha}(l)^{1-\alpha}+ (1-\delta)k-k')^{1-\sigma}}{1-\sigma}-\chi \frac{(l)^{1+\eta}}{1+\eta}$ is bounded, so $Tv$ is also bounded.\\
- Next, consider monotonicity. Suppose $v(k)\leq w(k)$ for all $k$. Let $\overline{g}(k)$be Howard's fixed policy function. 
\begin{equation}
\begin{split}
T^Hv(k) &=  U(\overline{g}(k))+\beta v(\overline{g}(k))\\
&\leq U(\overline{g}(k))+\beta w(\overline{g}(k))\\
&= T^Hw(k)
\end{split}
\end{equation}
- Lastly, consider discounting.
\begin{equation}
\begin{split}
T^H(v+a)(k) &=  U(\overline{g}(k))+\beta (v(\overline{g}(k))+a)\\
&= U(\overline{g}(k))+\beta v(\overline{g}(k))+\beta a\\
&= T^Hv(k) + \beta a
\end{split}
\end{equation}
The mapping used in Howard's fixed policy function satisfies Blackwell's theorem, therefore it is a contraction mapping.


\end{document}
